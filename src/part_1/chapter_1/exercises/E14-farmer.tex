Consider a farmer who produces wheat, corn, and sugar beets on his 500 acres of land. During the winter, the farmer wants to decide 
how much land to devote to each crop. 

The farmer knows that at least 200 tons (T) of wheat and 240T of corn are needed for cattle feed. 
These amounts can be raised on the farm or bought from a wholesaler. Any production in excess of the feeding requirement would be sold. Over the last decade, mean selling prices have been \$ 170 and \$ 150 per ton of wheat and corn, respectively. The purchase prices are 40 \% more than this due to the wholesaler's margin and transportation costs. The planting costs per acre of wheat and corn are \$ 150 and \$ 230, respectively.

Another profitable crop is sugar beet, which he expects to sell at \$36/T; however, the European Commission imposes a quota on sugar beet 
production. Any amount in excess of the quota can be sold only at \$10/T. The farmer’s quota for next year is 6000T. The planting cost per acre of sugar beet is \$ 260.

Based on past experience, the farmer knows that the mean yield on his land is
roughly 2.5T, 3T, and 20T per acre for wheat, corn, and sugar beets, respectively.

Based on the data, build up a model to help the farmer allocate the farming area to each crop and how much to sell/buy of wheat, corn, 
and sugar beets considering the following cases.

\begin{itemize}
	\item[(a)] The predictions are 100\% accurate and the mean yields are the only realizations possible.	
	\item[(b)] There are three possible equiprobable scenarios (i.e, each one with a probability equal to $\frac{1}{3}$): a good, fair, and bad weather scenario. In the good weather, the yield is 20\% better than the yield expected whereas in the bad weather scenario it is reduced 20\% of the mean yield. In the regular weather scenario, the yield for each crop keeps the historical mean - 2.5T/acre, 3T/acre, and 20T/acre for wheat, corn, and sugar beets, respectively.
	\item[(c)]	What happens if we assume the same scenarios as item (b) but with probabilities 25\%, 25\%, and 50\% for good, fair, and bad weather, respectively? How the production plan changes and why?	
\end{itemize}
