Consider the simplex method applied to a standard form minimization problem, and assume that the rows of the matrix $A$ are linearly independent. For each of the statements that follow, give either a proof or a counter example.
\begin{itemize}
\item[(a)] An iteration of the simplex method might change the feasible solution while leaving the cost unchanged.
%\item[(b)] A variable that has just left the basis cannot re-enter in the very next iteration.
\item[(b)] A variable that has just entered the basis cannot leave in the very next iteration.
\item[(c)] If there is a non-degenerate optimal basis, then there exists a unique optimal basis.
%\item[(e)] If $\bf x$ is an optimal solution found by the simplex method, no more than $m$ of its components can be positive, where $m$ is the number of equality constraints.
\end{itemize}