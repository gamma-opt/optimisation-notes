Recall the transportation problem from Chapter 1. Answer the following questions based on the interpretation of the dual price. 


We would like to plan the production and distribution of a certain product, taking into account that the transportation cost is known (e.g., proportional to the distance travelled), the factories (or source nodes) have a supply capacity limit, and the clients (or demand nodes) have known demands. Table \ref{p1c5:tab:E51_transport_problem_data} presents the data related to the problem. 

%
\begin{table}[h]
	\begin{tabular}{r|ccc|c}
    	& & {\it Clients} &\\\hline
    	{\it Factory} & NY & Chicago & Miami & Capacity \\\hline
    	Seattle & 2.5      & 1.7    & 1.8   & 350 \\
    	San Diego & 3.5 & 1.8 & 1.4 & 600 \\\hline
    	Demands & 325 & 300 & 275 & - \\\hline
	\end{tabular}
	\caption{Problem data: unit transportation costs, demands and capacities} \label{p1c5:tab:E51_transport_problem_data}
\end{table}
%
Additionally, we consider that the arcs (routes) from factories to clients have a maximum transportation capacity, assumed to be 250 units for each arc. The problem formulation is then 
%
\begin{align*}
	\mini z = \ &\sum_{i \in I}\sum_{j \in J}c_{ij}x_{ij} \\
	\st & \sum_{j \in J} x_{ij} \leq C_i, ~\forall i \in I \\
	& \sum_{i \in I} x_{ij} \geq D_j, ~\forall j \in J \\
	& x_{ij} \leq A_{ij}, ~\forall i \in I, j \in J \\
	& x_{ij} \geq 0, \forall i \in I, j \in J,
\end{align*}
%
where $C_i$ is the supply capacity of factory $i$, $D_j$ is the demand of client $j$ and $A_{ij}$ is the transportation capacity of the arc between $i$ and $j$.

\begin{itemize}
	\item[(a)] What price would the company be willing to pay for increasing the supply capacity of a given factory?
	\item[(b)] What price would the company be willing to pay for increasing the transportation capacity of a given arc?
\end{itemize}