\begin{itemize}
	\item[a)] According to Theorem \ref{p1c2:thm:fundamental_linear_algebra}, if the columns (or rows) of the $m \times m$ matrix $A$ are linearly independent, the system $Ax=b$ has a unique solution for every vector $b$. Explain the significance of this result in the context of linear optimization.
	\item[b)] If the columns of $A$ are not linearly independent, the system $Ax = b$ has either no solution or infinitely many solutions. Using the notions of \emph{span} and \emph{proper subspace}, explain why this is the case. 
	\item[c)] Solve the following systems of equations. Based on your findings, which systems have an invertible coefficient matrix?
	\begin{itemize}
		\item[] \begin{align*}
			2x_1 + 3x_2 + x_3 &= 12\\
			x_1 + 2x_2 + 3x_3 &= 12\\
			3x_1 + x_2 + 2x_3 &= 12
		\end{align*}
		\item[] \begin{align*}
			x_1 + 2x_2 + 4x_3 &= 6\\
			2x_1 + x_2 + 5x_3 &= 6\\
			x_1 + 3x_2 + 5x_3 &= 6
		\end{align*}
		\item[] \begin{align*}
			x_1 + 2x_2 + 4x_3 &= 5\\
			2x_1 + x_2 + 5x_3 &= 4\\
			x_1 + 3x_2 + 5x_3 &= 7
		\end{align*}
	\end{itemize}
\end{itemize}
