Consider the following integer programming problem $IP$:

\begin{table}[H]
	\centering
	\begin{tabular}{V{0.4cm} r V{0.1cm} r V{0.1cm} l}
		$\text{(IP)}$  & \multicolumn{4}{l}{z = max \ $x_1\ +\ 2x_2$}  \vspace{10pt}\\ 
		        $\st$  & $-\ 3x_1$ & $+$ & 4$x_2$ & $\leq$ &  4                     \\
		               &    3$x_1$ & $+$ & 2$x_2$ & $\leq$ & 11                     \\
		               &    2$x_1$ & $-$ &  $x_2$ & $\leq$ &  5        \vspace{10pt}\\
		               & \multicolumn{2}{l}{$x_1, x_2 \in \integers_+$}   
	\end{tabular}
\end{table}

Plot (or draw) the feasible region of the linear programming (LP) relaxation of the problem $IP$, then solve the problems using the figure. Recall that the LP relaxation of $IP$ is obtained by replacing the integrality constraints $x_1,x_2\in \integers_+$ by linear nonnegativity $x_1,x_2\geq 0$ and upper bounds corresponding to the upper bounds of the integer variables ($x_1,x_2\leq 1$ for binary variables). 

\begin{itemize}
	\item[(a)] What is the optimal cost $z_{LP}$ of the LP relaxation of the problem $IP$? What is the optimal cost $z$ of the problem $IP$?
	\item[(b)] Draw the border of the convex hull of the feasible solutions of the problem $IP$. Recall that the convex hull represents the \emph{ideal} formulation for the problem $IP$.
	\item[(c)] Solve the problem $IP$ by LP-relaxation based branch-and-bound. You can solve the LP relaxations at each node of the branch-and-bound tree graphically. Start the branch-and-bound procedure without any primal bound.
\end{itemize}
