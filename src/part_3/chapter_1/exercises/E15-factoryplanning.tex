A factory makes seven products (PROD 1 to PROD 7) using the
following machines: four grinders, two vertical drills, three horizontal drills, one
borer and one planer. Each product yields a certain contribution to the profit (defined
as \$/unit selling price minus the cost of raw materials). These quantities (in \$/unit)
together with the unit production times (hours) required on each process are given
in Table \ref{p1c1:tab:ex1-5_prod_yield}. A dash indicates that a product does not require a process. 
There are also marketing demand limitations on each product each month. These are given in Table \ref{p1c1:tab:ex1-5_max_demand}.

\begin{table}[h]
	\begin{tabular}{l|lllllll}
		& \textbf{PROD1} & \textbf{PROD2} & \textbf{PROD3} & \textbf{PROD4} & \textbf{PROD5} & \textbf{PROD6} & \textbf{PROD7} \\ \hline 
		\textbf{Profit} & 10    & 6     & 8     & 4     & 11    & 9     & 3     \\
		\textbf{Grinding}              & 1.5   & 2.1   & –     & –     & 0.9   & 0.6   & 1.5   \\
		\textbf{Vert. drilling}     & 0.3   & 0.6   & –     & 0.9   & –     & 1.8   & –     \\
		\textbf{Horiz. drilling}   & 0.6   & –     & 2.4   & –     & –     & –     & 1.8   \\
		\textbf{Boring}                & 0.15  & 0.09  & –     & 0.21  & 0.3   & –     & 0.24  \\
		\textbf{Planing}               & –     & –     & 0.03  & –     & 0.15  & –     & 0.15 \\ \hline
	\end{tabular}
	\caption{Product yields}
	\label{p1c1:tab:ex1-5_prod_yield}
\end{table}

\begin{table}[H]
	\begin{tabular}{l|lllllll}
		& \textbf{PROD1} & \textbf{PROD2} & \textbf{PROD3} & \textbf{PROD4} & \textbf{PROD5} & \textbf{PROD6} & \textbf{PROD7} \\ \hline
		\textbf{January}   & 500   & 1000  & 300   & 300   & 800   & 200   & 100   \\
		\textbf{February}  & 600   & 500   & 200   & 0     & 400   & 300   & 150   \\
		\textbf{March}     & 300   & 600   & 0     & 0     & 500   & 400   & 100   \\
		\textbf{April}     & 200   & 300   & 400   & 500   & 200   & 0     & 100   \\
		\textbf{May}       & 0     & 100   & 500   & 100   & 1000  & 300   & 0     \\
		\textbf{June}      & 500   & 500   & 100   & 300   & 1100  & 500   & 60    \\ \hline
	\end{tabular}
	\caption{Maximum demand}
	\label{p1c1:tab:ex1-5_max_demand}
\end{table}

It is possible to store up to 100 of each product at a time at a cost of \$0.5 per unit per month. There are no stocks at present, but it is desired to have a stock of 50 of each type of product at the end of June. 

The factory works six days a week with two shifts of 8h each day. Assume that each month consists of only 24 working days. Also, there are no penalties for unmet demands. What is the factory's production plan (how much of which product to make and when) in order to maximise the total profit?

